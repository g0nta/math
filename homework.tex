\documentclass{jsarticle}
\usepackage{amsthm}
\usepackage{amsmath}


\newtheorem{Theorem}{定理}
\newtheorem{Definition}{定義}

\newcommand{\Pow}{ {\rm Pow} }
\newcommand{\Map}{ {\rm Map} }
\newcommand{\2}{ {\bf 2} }

\begin{document}
\section{定義の確認}
本節では後の証明のため、定義の確認を行う。
$V,W$を集合、$f$を$V$から$W$への写像とする。
\begin{Definition}[関数集合]
    $\Map(V,W) = W^V = \{f | f:V \rightarrow W\}$.
\end{Definition}

\begin{Definition}[冪集合]
    $\Pow(V) = \{S | S \subseteq V \}$.
\end{Definition}

\begin{Definition}[可逆性]
    $f(v)=w \Leftrightarrow f^{-1}(w)=v$を満たす写像$f^{-1}$が存在するとき、$f$は{\bf 可逆}であるという。
\end{Definition}

\begin{Definition}[単射]
    $\forall v_1,v_2 \in V$に対して、$u \neq v \Rightarrow f(u) \neq f(v)$を満たす時、$f$は{\bf 単射}であるという。
\end{Definition}

\begin{Definition}[全射]
    $\forall w \in W$に対して、ある$v \in V$が存在して、$f(v)=w$を満たす時、$f$は{\bf 全射}であるという。
\end{Definition}

\begin{Definition}[双射]
    $f$が全射かつ単射であるとき、$f$は{\bf 双射}であるという。
\end{Definition}

\begin{Definition}[集合としての同型]
    集合$A,B$の間に全単射が存在するとき、$A,B$は{\bf 同型}であるといい、$A \simeq B$と表記する。
\end{Definition}

\begin{Definition}[特殊な集合]
    $\2 = \{0,1\}$.
\end{Definition}

\section{定理と証明}
\begin{Theorem}[可逆生と双射性]
    $f$が可逆$\Leftrightarrow$ $f$は双射
\end{Theorem}
\begin{proof}[証明]
    まず、{\bf $f$が双射$\Rightarrow$ $f$が可逆}を示す。
    
    任意の$v\in V$に対して、$f(v)=w$とすると、$f$は単射なので、$w$の逆元は一意に定まる。
    よって
    \begin{align*}
        f(v)=w \Rightarrow f^{-1}(w)=v.
    \end{align*}
    また、$f$は全射でもあるので、任意の$w \in W$に対して逆元が一意に定まる。よって
    \begin{align*}
        f(v)=w \Leftarrow f^{-1}(w)=v.
    \end{align*}
    よって、$f$が双射であるとき、$f$は可逆。

    次に、{\bf $f$が可逆 $\Rightarrow$ $f$が双射}
    を示す。これを示すために
    (i){\bf $f$が可逆 $\Rightarrow$ $f$が単射}
    と
    (ii){\bf $f$が可逆 $\Rightarrow$ $f$が全射}
    を示す。\\
    (i)対偶を示す。$f$が単射でないとする。すなわち$f$は
    \begin{align*}
        \exists v_1, v_2 \in V, v_1 \neq v_2 \land f(v_1)=f(v_2).
    \end{align*}
    を満たすとする。また、$w_0 \in W$を$w_0 = f(v_1)=f(v_2)$とする。
    $f$が可逆であるためには
    $f(v_1)=w_0 \Leftrightarrow f^{-1}(w_0)=v_1$ かつ $f(v_2)=w_0 \Leftrightarrow f^{-1}(w_0)=v_2$
    でなければならないが、これは$f^{-1}$が写像であることに反する。よって$f$は可逆でない。\\
    (ii)対偶を示す。$f$が全射でないとする。
    $f$は全射でないので、ある$w_0 \in W$が存在して、任意の$v \in V$に対して、$f(v) \neq w_0$を満たすとする。
    この$w_0$には$f$の逆像が存在しないので、$f$は可逆ではない。
\end{proof}

\begin{Theorem}
    $A$を集合とする。$\Pow(A) \simeq {\2}^A$
\end{Theorem}
\begin{proof}
    $g \in {\2}^A$とする。$g$から$A$の冪集合$S_g$への写像$h$を以下のように定義すれば$h$が双射であることは明らか。
    \begin{align*}
        h(g) = \{ a | g(a)=1 \}.
    \end{align*}
\end{proof}
\end{document}