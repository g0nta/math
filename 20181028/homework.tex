\documentclass{jsarticle}
    \usepackage{amsthm}
    \usepackage{amsmath}
    
    \newtheorem{Theorem}{定理}
    \newtheorem{Definition}{定義}
    
    \newcommand{\Pow}{ {\rm Pow} }
    \newcommand{\Map}{ {\rm Map} }
    \newcommand{\2}{ {\bf 2} }
    
    \begin{document}
    \section{定義の確認}
    本節では後の証明のため、定義の確認を行う。
    $V,W$を集合、$f$を$V$から$W$への写像とする。
    \begin{Definition}[関数集合]
        $\Map(V,W) = W^V = \{f | f:V \rightarrow W\}$.
    \end{Definition}
    
    \begin{Definition}[冪集合]
        $\Pow(V) = \{S | S \subseteq V \}$.
    \end{Definition}
    
    \begin{Definition}[可逆性]
        $f(v)=w \Leftrightarrow f^{-1}(w)=v$を満たす写像$f^{-1}$が存在するとき、$f$は{\bf 可逆}であるという。
    \end{Definition}
    
    \begin{Definition}[単射]
        $\forall v_1,v_2 \in V$に対して、$u \neq v \Rightarrow f(u) \neq f(v)$を満たす時、$f$は{\bf 単射}であるという。
    \end{Definition}
    
    \begin{Definition}[全射]
        $\forall w \in W$に対して、ある$v \in V$が存在して、$f(v)=w$を満たす時、$f$は{\bf 全射}であるという。
    \end{Definition}
    
    \begin{Definition}[双射]
        $f$が全射かつ単射であるとき、$f$は{\bf 双射}であるという。
    \end{Definition}
    
    \begin{Definition}[集合としての同型]
        集合$A,B$の間に全単射が存在するとき、$A,B$は{\bf 同型}であるといい、$A \simeq B$と表記する。
    \end{Definition}
    
    \begin{Definition}[特殊な集合]
        $\2 = \{0,1\}$.
    \end{Definition}
    
    \section{定理と証明}
    \begin{Theorem}
        $A$を集合とする。$\Pow(A) \simeq {\2}^A$
    \end{Theorem}
    \begin{proof}
        $g \in {\2}^A$とする。
        $g$から$A$の冪集合$\Pow(A)$への写像$h$を以下のように定義する。
        \begin{align*}
            h(g) = \{ a | g(a)=1 \}.
        \end{align*}
        $h$が双射であることを示す。

        (i){\bf $h$が単射} \\
        $\forall g,g' \in \2^A$とする。$g \neq g' \rightarrow h(g) \neq h(g')$を示す。
        $g \neq g'$なので、ある$a_0 \in A$が存在して、$g(a_0) \neq g'(a_0)$を満たす。
        $g(a_0) = 1$としても一般性を失わない。このとき、$a_0 \in h(g)$だが、$a_0 \in h(g')$である。
        よって、$h(g) \neq h(g')$.

        (ii){\bf $h$が全射} \\
        任意の$S \in \Pow(A)$から、$g_S \in \2^A$を構成できれば十分である。
        $g_S$を下記のように定義する。
        \begin{align*}
            g_S(a) = \begin{cases}
                1 & (a \in S) \\
                0 & (a \notin S)
            \end{cases}
        \end{align*}
        明らかに$g_S \in \2^A$である。
    \end{proof}
    \end{document}