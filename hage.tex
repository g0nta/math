\documentclass{jsarticle}
\usepackage{amsthm}
\usepackage{amsmath}


\newtheorem{Theorem}{定理}
\newtheorem{Definition}{定義}

\begin{document}
\section{定義の確認}
本節では後の証明のため、定義の確認を行う。
$V,W$を集合、$f$を$V$から$W$への写像とする。
\begin{Definition}[可逆性]
    $f(v)=w \Leftrightarrow f^{-1}(w)=v$を満たす写像$f^{-1}$が存在するとき、$f$は{\bf 可逆}であるという。
\end{Definition}

\begin{Definition}[単射]
    $\forall v_1,v_2 \in V$に対して、$u \neq v \Rightarrow f(u) \neq f(v)$を満たす時、$f$は{\bf 単射}であるという。
\end{Definition}

\begin{Definition}[全射]
    $\forall w \in W$に対して、ある$v \in V$が存在して、$f(v)=w$を満たす時、$f$は{\bf 全射}であるという。
\end{Definition}

\begin{Definition}[双射]
    $f$が全射かつ単射であるとき、$f$は{\bf 双射}であるという。
\end{Definition}

\begin{Definition}[特殊な集合]
    ${\bf 2} = \{0,1\}$.
\end{Definition}

\section{定理と証明}
\begin{Theorem}[可逆生と双射性]
    $f$が可逆$\Leftrightarrow$ $f$は双射
\end{Theorem}
\begin{proof}[証明]
    まず、$f$が可逆 $\Rightarrow$ $f$が双射を示す。

    (i)$f$が可逆 $\Rightarrow$ $f$が単射

    と

    (ii)$f$が可逆 $\Rightarrow$ $f$が全射

    を示す。

    (i)対偶を示す。$f$が単射でないとする。すなわち$f$は
    \begin{align}
        \exists v_1, v_2 \in V, v_1 \neq v_2 \land f(v_1)=f(v_2).
    \end{align}
    を満たすとする。また、$w \in W$を$w = f(v_1)=f(v_2)$とする。
\end{proof}

\end{document}